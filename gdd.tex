\documentclass[11pt, a4paper]{article}
\usepackage[utf8]{inputenc}
\usepackage[spanish]{babel}
\usepackage{geometry}
\usepackage{graphicx}
\usepackage{hyperref}
\usepackage{enumitem}
\usepackage{listings}
\usepackage{xcolor}
\usepackage{amsmath}

\geometry{left=2.5cm, right=2.5cm, top=2.5cm, bottom=2.5cm}

\title{\textbf{Documento de Diseño Técnico (GDD)} \\ Proyecto: "Simulador de Vida - Efectos del Tabaquismo"}
\author{Departamento de Prevención de Adicciones}
\date{\today}

\begin{document}

\maketitle

\section{Visión General del Proyecto}
El objetivo es desarrollar una aplicación web móvil (Web App) gamificada estilo "Mascota Virtual" (similar a Pou o Tamagotchi). El usuario cuidará a un personaje al que puede alimentar con productos nocivos (tabaco) o saludables.
\\
\\
\textbf{Objetivo Pedagógico:} Demostrar de forma interactiva y acelerada el deterioro físico y la pérdida de control (adicción) causados por el tabaquismo, así como los beneficios de la cesación.
\\
\textbf{Acceso:} Vía escaneo de código QR (sin instalación de Apps).

\section{Requerimientos Técnicos}
\begin{itemize}
    \item \textbf{Plataforma:} Web App (HTML5, CSS3, JavaScript Vanilla).
    \item \textbf{Diseño:} Mobile-First (Diseño vertical responsivo).
    \item \textbf{Persistencia:} Uso de \texttt{localStorage} del navegador. La muerte o enfermedad es persistente.
    \item \textbf{Peso:} Optimización máxima de assets para carga instantánea.
\end{itemize}

\section{Mecánicas y Lógica del Juego}

\subsection{Variables de Estado}
\begin{enumerate}
    \item \textbf{Salud ($S$):} Inicia en 100\%. Representa la integridad sistémica (Pulmonar, Cardíaca, Inmunológica).
    \item \textbf{Nivel de Adicción ($A$):} Inicia en 0\%. Modifica la probabilidad de éxito de acciones saludables.
    \item \textbf{Tiempo desde el último cigarro ($T_{off}$):} Variable oculta que activa la lógica de recuperación.
\end{enumerate}

\subsection{Algoritmo de Interacción}

\subsubsection{Acción: "Fumar / Vapear"}
\begin{itemize}
    \item \textbf{Lógica:} $S_{nuevo} = S_{actual} - 15$ ; $A_{nuevo} = A_{actual} + 20$.
    \item \textbf{Efecto Inmediato:} Vasoconstricción (pulsación rápida de pantalla), toses esporádicas.
    \item \textbf{Reinicio de $T_{off}$:} Se pone a cero.
\end{itemize}

\subsubsection{Acción: "Cuidado Saludable" (Comida/Deporte)}
\begin{itemize}
    \item \textbf{La Barrera de la Abstinencia:} Si $A > 40\%$, hay un factor de rechazo $R$:
    $$ P(Rechazo) = \min(A, 90\%) $$
    Si se rechaza, el personaje pide tabaco y la salud \textbf{no sube}.
    \item \textbf{Recuperación Activa:} Si se acepta: $S = S + 5, A = A - 5$.
\end{itemize}

\subsubsection{Mecánica de Recuperación Pasiva ($T_{off}$)}
Si el usuario deja de fumar, el sistema aplica beneficios automáticos basados en el tiempo real (escalado a tiempo de juego):
\begin{itemize}
    \item \textbf{Modo 12h:} Normalización de niveles de monóxido de carbono (Sube $S$ un 2\%).
    \item \textbf{Modo 48h:} Mejora de olfato/gusto (El personaje sonríe más al comer sano).
    \item \textbf{Modo 1 año:} Riesgo cardíaco se reduce al 50\%.
\end{itemize}

\section{Fisiopatología en el Personaje (Evolución Visual)}

\begin{center}
\begin{tabular}{|c|c|l|l|}
\hline
\textbf{Estado} & \textbf{Salud ($S$)} & \textbf{Síntomas Visuales} & \textbf{Correlación Médica} \\
\hline
1. Sano & $100-81\%$ & Rosado, ojos claros. & Estado basal saludable. \\
\hline
2. Inicial & $80-51\%$ & Dientes amarillos, halitosis. & Tinción por alquitrán. \\
\hline
3. Deterioro & $50-21\%$ & Piel grisácea, tos crónica. & Hipoxia tisular periférica. \\
\hline
4. Crítico & $20-1\%$ & Cianosis (labios azules), O2. & EPOC avanzada / Enfisema. \\
\hline
5. Final & $0\%$ & Lápida / Historial Clínico. & Colapso multisistémico. \\
\hline
\end{tabular}
\end{center}

\section{Detalle de Enfermedades Implementadas}
\begin{itemize}
    \item \textbf{EPOC (Enfermedad Pulmonar Obstructiva Crónica):} Se manifiesta reduciendo la velocidad de movimiento del personaje. No es reversible al 100\%, dejando una "secuela" en la salud máxima ($S_{max} = 85\%$).
    \item \textbf{Isquemia Periférica:} El personaje cambia de color a un tono cenizo debido a la reducción del flujo sanguíneo por la nicotina.
    \item \textbf{Síndrome de Abstinencia:} El "temblor" del personaje cuando $T_{off} > 0$ pero $A$ es alto. Esto genera ansiedad visual en el usuario.
\end{itemize}

\section{Proceso de Curación (Línea de Tiempo Pedagógica)}
El simulador mostrará mensajes informativos según el tiempo de "no fumar":
\begin{enumerate}
    \item \textbf{20 minutos:} El pulso y presión arterial bajan a niveles normales.
    \item \textbf{12 horas:} El monóxido de carbono en sangre baja a niveles normales.
    \item \textbf{48 horas:} Terminaciones nerviosas se regeneran; mejora gusto y olfato.
    \item \textbf{2 semanas - 3 meses:} La función pulmonar aumenta hasta un 30\%.
    \item \textbf{1 - 9 meses:} Disminuye la tos y la falta de aire; los cilios pulmonares recuperan función.
\end{enumerate}

\section{Interfaz y Call to Action}
Cuando $S = 0$, se muestra un diagnóstico basado en el historial:
\begin{itemize}
    \item "Muerte por Cáncer de Pulmón (90\% de los casos son por fumar)".
    \item "Colapso por EPOC: Tus pulmones perdieron la elasticidad".
\end{itemize}
\textbf{CTA:} Botón de ayuda redirigiendo a: \url{https://www.who.int/es/news-room/fact-sheets/detail/tobacco}

\end{document}
